\subsection{Question 5}

	In equation 3 of the instructions, the prediction step is given by equation \ref{eq:part_iii_prediction3}, 
	while the update step is given by \ref{eq:part_iii_update3}.
	
	\begin{equation}
		\overline{bel}(x_t) = p(x_t | u_{1:t}, z_{1:t-1}, \bar{x}_0, M) = \int p(x_t | u_t, x_{t-1}) bel(x_{t-1}) dx_{t-1}
		\label{eq:part_iii_prediction3}
	\end{equation}
	
	\begin{equation}
		bel(x_t) = p(x_t | u_{1:t}, z_{1:t}, \bar{x}_0, M) = \eta p(z_t | x_t, M) \overline{bel}(x_t)
		\label{eq:part_iii_update3}
	\end{equation}
	
	Since the prediction step at time $t$ refers to the belief about the state at time $t$ before incorporating the
	measurements $z_t$, with regard to equation 2 in the instructions, the prediction step is given by factor in
	\ref{eq:part_iii_prediction2}, while the update step is given by the factor in \ref{eq:part_iii_update2}.
	
	\begin{equation}
		\int p(x_t | u_t, x_{t-1}) p(x_{t-1} | z_{1:t-1}, u_{1:t-1}, \bar{x}_0, M)) dx_{t-1}
		\label{eq:part_iii_prediction2}
	\end{equation}
	
	\begin{equation}
		\eta p(z_t | x_t, M) \int p(x_t | u_t, x_{t-1}) p(x_{t-1} | z_{1:t-1}, u_{1:t-1}, \bar{x}_0, M)) dx_{t-1}
		\label{eq:part_iii_update2}
	\end{equation}
	
	
\subsection{Question 6}

	The measurements are independent of each other if-f the measurement noise is white Gaussian noise, which is in fact assumed here,
	as the value of the Jacobian $H$ depends only on the value of $\hat{x}_t$ at time $t$. 
	Then, the covariance matrix of the measurement noise distribution will be diagonal and all measurements will be uncorrelated
	with each other.
	
	
\subsection{Question 7}

	Since $\delta_M$ is a probability, $0 \leq \delta_M \leq 1$. The inverse $\chi^2$ is monotonously ascending, hence a larger value for $\delta_M$
	will result in a larger $\lambda_M$ threshold. The larger the threshold, the larger the Mahalanobis distance 
	$D_M = (\overline{\nu}_t^i)^T (H_{t,j} \bar{\Sigma}_t (H_{t,j})^T + Q) (\overline{\nu}_t^i)$, which means that the smaller the uncertainty 
	in measurement $Q$ is. Hence, a large value for $\delta_M$ will result in having more confidence on the measurements and, hence, less outlier rejections.
	Conversely, a low value for $\delta_M$ will result in more rejections. As for the threshold $\lambda_M$, it follows the same reasoning. If we know that
	the measurements are reliable, then $Q$ is small, and we can utilize a higher threshold. On the other hand, if the measurements are known to be spurious,
	then a lower $\lambda_M$ threshold should be used, so as to take into account only the most reliable ones, which are the ones closest to the centroid of 
	the Mahalanobis ellipse.
	
	
\subsection{Question 8}
	
	If the first measurement to be incorporated is noisy, then the innovation will be non-zero and the estimated mean for that measurement will be shifted
	erroneously. The covariance $\bar{\Sigma}$ will be incorrectly reduced, which will result in a higher Kalman gain in the next timestep. Then,
	$S_{t,j}$ will be reduced, and the threshold on the Mahalanobis distance will increase, resulting in possibly valid measurements being rejected as outliers.
	

\subsection{Question 9}

	In Alg 4 of the instructions, the first three assignments are not dependant on the observation $i$, however each one is computed $|z_t| \times |M|$ times,
	where $|z_t|$ is the number of measured parameters and $|M|$ the number of landmarks. Hence, these computations are redundant and could be made 
	beforehand in a separate loop, so as to reduce $|z_t| \times |M|$ to just $|M|$.
	

\subsection{Question 10}

	Matrix $\mathbf{h_t}$ is of $2 \times 1$ dimensions. Hence, the measurement matrix $\mathbf{z_t}$ will also be of $2 \times 1$ dimensions. The same applies for the
	innovation $\bar{\nu}_t^k$. $(\bar{\nu}_t^k)^T$ is of $1 \times 2$ dimensions, and $n$ of those vectors are stored in $(\bar{\nu}_t)^T$. Hence, $\bar{\nu}_t$ is of
	$2n \times 1$ dimensions, where $n$ is the number of inliers. Analogously, $\bar{H}_t$ is of $2n \times 3$ dimensions. In the sequential update method, 
	$\bar{\nu}_t$ and $\bar{H}_t$ have $2 \times 1$ and $2 \times 3$ dimensions respectively. This is to be expected, since in the batch update method all the 
	observations are grouped together, whereas in the sequential one, each observation is treated separately. The higher dimensionality in the former case
	will result in more computational costs than in the latter.